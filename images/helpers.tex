%\newcommand{\UP}{1}
%\newcommand{\RIGHT}{2}
%\newcommand{\DOWN}{3}
%\newcommand{\LEFT}{4}

% (x, y) denotes the top left position of the arrow
% length, x, y, orientation
%\newcommand{\spin}[4][0.5]{
%    \ifthenelse{#4 = \UP}{
%        \draw[->] (#2, #3) -- (#2, #3 + #1);
%    }{}
%    \ifthenelse{#4 = \RIGHT}{
%        \draw[->] (#2, #3) -- (#2 + #1, #3);
%    }{}
%    
%    \ifthenelse{#4 = \DOWN}{
%        \draw[->] (#2, #3) -- (#2, #3);
%    }{}
%}
\newcommand{\up}[0]{$\uparrow$}
\newcommand{\down}[0]{$\downarrow$}

\newcommand{\rup}[0]{\textcolor{red}{\up}}
\newcommand{\rdown}[0]{\textcolor{red}{\down}}

\newcommand{\gup}[0]{\textcolor{green}{\up}}
\newcommand{\gdown}[0]{\textcolor{green}{\down}}

\newcommand{\ising}[0]{
    \begin{equation*}
        H = - J \sum_{<i, j>} S_i S_j
    \end{equation*}
}

\newenvironment{mybox}{\begin{tcolorbox}[sharp corners=all, frame empty]}{\end{tcolorbox}}
