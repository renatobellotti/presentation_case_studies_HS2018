% Renato Bellotti, October 2018

\documentclass{beamer}

\usepackage{tikz}
%\usepackage{enumitem}
\usepackage{tcolorbox}
\usepackage{amsmath}

\setbeamertemplate{navigation symbols}{}

%\newcommand{\UP}{1}
%\newcommand{\RIGHT}{2}
%\newcommand{\DOWN}{3}
%\newcommand{\LEFT}{4}

% (x, y) denotes the top left position of the arrow
% length, x, y, orientation
%\newcommand{\spin}[4][0.5]{
%    \ifthenelse{#4 = \UP}{
%        \draw[->] (#2, #3) -- (#2, #3 + #1);
%    }{}
%    \ifthenelse{#4 = \RIGHT}{
%        \draw[->] (#2, #3) -- (#2 + #1, #3);
%    }{}
%    
%    \ifthenelse{#4 = \DOWN}{
%        \draw[->] (#2, #3) -- (#2, #3);
%    }{}
%}
\newcommand{\up}[0]{$\uparrow$}
\newcommand{\down}[0]{$\downarrow$}

\newcommand{\rup}[0]{\textcolor{red}{\up}}
\newcommand{\rdown}[0]{\textcolor{red}{\down}}

\newcommand{\gup}[0]{\textcolor{green}{\up}}
\newcommand{\gdown}[0]{\textcolor{green}{\down}}

\newcommand{\ising}[0]{
    \begin{equation*}
        H = - J \sum_{<i, j>} S_i S_j
    \end{equation*}
}

\newcommand*\circled[1]{\tikz[baseline=(char.base)]{
            \node[shape=circle,draw,inner sep=2pt] (char) {#1};}}

\newcommand*\rcircled[1]{\tikz[baseline=(char.base)]{
            \node[shape=circle,draw,inner sep=2pt, color=red] (char) {#1};}}

\newenvironment{mybox}{\begin{tcolorbox}[sharp corners=all, frame empty]}{\end{tcolorbox}}


\begin{document}

\begin{frame}{Multi-GPU accelerated multi-spin Monte Carlo simulations of the 2D Ising model}
\begin{tabular}{c c c}
    %\begin{tikzpicture}
%\spin{1}{1}{2}
%\end{tikzpicture}
%\documentclass{article}

%\usepackage{fancyhdr}

%\pagestyle{fancy}
%\fancyfoot{}

%%\newcommand{\UP}{1}
%\newcommand{\RIGHT}{2}
%\newcommand{\DOWN}{3}
%\newcommand{\LEFT}{4}

% (x, y) denotes the top left position of the arrow
% length, x, y, orientation
%\newcommand{\spin}[4][0.5]{
%    \ifthenelse{#4 = \UP}{
%        \draw[->] (#2, #3) -- (#2, #3 + #1);
%    }{}
%    \ifthenelse{#4 = \RIGHT}{
%        \draw[->] (#2, #3) -- (#2 + #1, #3);
%    }{}
%    
%    \ifthenelse{#4 = \DOWN}{
%        \draw[->] (#2, #3) -- (#2, #3);
%    }{}
%}
\newcommand{\up}[0]{$\uparrow$}
\newcommand{\down}[0]{$\downarrow$}

\newcommand{\rup}[0]{\textcolor{red}{\up}}
\newcommand{\rdown}[0]{\textcolor{red}{\down}}

\newcommand{\gup}[0]{\textcolor{green}{\up}}
\newcommand{\gdown}[0]{\textcolor{green}{\down}}

\newcommand{\ising}[0]{
    \begin{equation*}
        H = - J \sum_{<i, j>} S_i S_j
    \end{equation*}
}

\newcommand*\circled[1]{\tikz[baseline=(char.base)]{
            \node[shape=circle,draw,inner sep=2pt] (char) {#1};}}

\newcommand*\rcircled[1]{\tikz[baseline=(char.base)]{
            \node[shape=circle,draw,inner sep=2pt, color=red] (char) {#1};}}

\newenvironment{mybox}{\begin{tcolorbox}[sharp corners=all, frame empty]}{\end{tcolorbox}}


%\begin{document}
\setlength{\tabcolsep}{8pt}
\renewcommand{\arraystretch}{1.2}

\begin{tabular}[b]{c c c c c}
    \up & \down & \down & \up & \down \\
    \down & \up & \down & \down & \down \\
    \up & \up & \down & \up & \down \\
    \up & \down & \up & \up & \down
    %$\uparrow$ & $\uparrow$ & $\uparrow$ & $\uparrow$ & $\uparrow$
\end{tabular}
%\end{document}

    \textbf{\huge{+}} &
    \includegraphics[keepaspectratio=true, width=0.2\paperwidth]{images/multi_gpu.png}
    \textbf{\huge{=}} &
    \includegraphics[keepaspectratio=true, width=0.2\paperwidth]{images/cheetah.jpg}
\end{tabular}
\end{frame}

\setcounter{framenumber}{0}
\setbeamertemplate{navigation symbols}{\insertframenumber}

\begin{frame}{What was done?}
    \begin{minipage}{0.5\paperwidth}
        \begin{itemize}
            \item \textbf{Ising model}
                \noindent \ising
            \item \textbf{Metropolis algorithm:}\\
                Efficient sampling
                \begin{equation*}
                    p_a = e^{-\frac{E}{k_B T}}
                \end{equation*}
        \end{itemize}
    \end{minipage} \pause
    \hfill
    \begin{minipage}[c]{0.3\paperwidth}
        \begin{itemize}%[itemsep=5mm]
            \item Single core CPU \pause
            \item Single GPU \pause
            \item Multpiple GPUs \pause
        \end{itemize}
    \end{minipage}
\begin{mybox}
    How does it scale? Is it worth the effort?
\end{mybox}
\end{frame}

\begin{frame}{Ising model I}
\begin{itemize}
    \item A standard model of statistical physics \pause
    \item classical model \pause
    \item Formula:
        \ising \pause
    \item Describes magnets:\\
        \begin{minipage}{0.3\paperwidth}%\begin{tikzpicture}
%\spin{1}{1}{2}
%\end{tikzpicture}
%\documentclass{article}

%\usepackage{fancyhdr}

%\pagestyle{fancy}
%\fancyfoot{}

%%\newcommand{\UP}{1}
%\newcommand{\RIGHT}{2}
%\newcommand{\DOWN}{3}
%\newcommand{\LEFT}{4}

% (x, y) denotes the top left position of the arrow
% length, x, y, orientation
%\newcommand{\spin}[4][0.5]{
%    \ifthenelse{#4 = \UP}{
%        \draw[->] (#2, #3) -- (#2, #3 + #1);
%    }{}
%    \ifthenelse{#4 = \RIGHT}{
%        \draw[->] (#2, #3) -- (#2 + #1, #3);
%    }{}
%    
%    \ifthenelse{#4 = \DOWN}{
%        \draw[->] (#2, #3) -- (#2, #3);
%    }{}
%}
\newcommand{\up}[0]{$\uparrow$}
\newcommand{\down}[0]{$\downarrow$}

\newcommand{\rup}[0]{\textcolor{red}{\up}}
\newcommand{\rdown}[0]{\textcolor{red}{\down}}

\newcommand{\gup}[0]{\textcolor{green}{\up}}
\newcommand{\gdown}[0]{\textcolor{green}{\down}}

\newcommand{\ising}[0]{
    \begin{equation*}
        H = - J \sum_{<i, j>} S_i S_j
    \end{equation*}
}

\newcommand*\circled[1]{\tikz[baseline=(char.base)]{
            \node[shape=circle,draw,inner sep=2pt] (char) {#1};}}

\newcommand*\rcircled[1]{\tikz[baseline=(char.base)]{
            \node[shape=circle,draw,inner sep=2pt, color=red] (char) {#1};}}

\newenvironment{mybox}{\begin{tcolorbox}[sharp corners=all, frame empty]}{\end{tcolorbox}}


%\begin{document}
\setlength{\tabcolsep}{8pt}
\renewcommand{\arraystretch}{1.2}

\begin{tabular}[b]{c c c c c}
    \up & \down & \down & \up & \down \\
    \down & \up & \down & \down & \down \\
    \up & \up & \down & \up & \down \\
    \up & \down & \up & \up & \down
    %$\uparrow$ & $\uparrow$ & $\uparrow$ & $\uparrow$ & $\uparrow$
\end{tabular}
%\end{document}
\end{minipage}
        \hfill
        \begin{minipage}{0.3\paperwidth}\includegraphics[keepaspectratio=true, width=0.3\paperwidth]{images/magnet.jpg}\end{minipage} \pause
    \item System sizes: $100'000 \times 100'000$
\end{itemize}
\end{frame}

\begin{frame}{Ising model II}
    Nearest neighbour interactions only!
    \ising \pause
    \vspace{1cm}
    \begin{minipage}{0.3\paperwidth}{
\setlength{\tabcolsep}{8pt}
\renewcommand{\arraystretch}{1.2}

\begin{tabular}[b]{c c c c c}
    \rup & \rup & \rup & \rup & \rdown \\
    \rdown & \rup & \gdown & \rdown & \rup \\
    \rup & \rdown & \rup & \rdown & \rup \\ 
\end{tabular}
}
\end{minipage}
    {\Huge$\Rightarrow$}
    \begin{minipage}{0.2\paperwidth}{
\setlength{\tabcolsep}{8pt}
\renewcommand{\arraystretch}{1.2}

\begin{tabular}[b]{c c c c c}
    \up & \up & \rup & \up & \down \\
    \down & \rup & \gdown & \rdown & \up \\
    \up & \down & \rup & \down & \up \\ 
\end{tabular}
}
\end{minipage} \pause
    
    Calculation of energy: $\mathcal{O}\left( n^2 \right) \Rightarrow \mathcal{O}\left( n \right)$
\end{frame}

\begin{frame}{Metropolis algorithm}
\begin{itemize}
    \item \textbf{Goal:} Sample phase space\\
        \vspace{3mm}
        \begin{minipage}[c]{0.4\textwidth}
            %\begin{tikzpicture}
%\spin{1}{1}{2}
%\end{tikzpicture}
%\documentclass{article}

%\usepackage{fancyhdr}

%\pagestyle{fancy}
%\fancyfoot{}

%%\newcommand{\UP}{1}
%\newcommand{\RIGHT}{2}
%\newcommand{\DOWN}{3}
%\newcommand{\LEFT}{4}

% (x, y) denotes the top left position of the arrow
% length, x, y, orientation
%\newcommand{\spin}[4][0.5]{
%    \ifthenelse{#4 = \UP}{
%        \draw[->] (#2, #3) -- (#2, #3 + #1);
%    }{}
%    \ifthenelse{#4 = \RIGHT}{
%        \draw[->] (#2, #3) -- (#2 + #1, #3);
%    }{}
%    
%    \ifthenelse{#4 = \DOWN}{
%        \draw[->] (#2, #3) -- (#2, #3);
%    }{}
%}
\newcommand{\up}[0]{$\uparrow$}
\newcommand{\down}[0]{$\downarrow$}

\newcommand{\rup}[0]{\textcolor{red}{\up}}
\newcommand{\rdown}[0]{\textcolor{red}{\down}}

\newcommand{\gup}[0]{\textcolor{green}{\up}}
\newcommand{\gdown}[0]{\textcolor{green}{\down}}

\newcommand{\ising}[0]{
    \begin{equation*}
        H = - J \sum_{<i, j>} S_i S_j
    \end{equation*}
}

\newcommand*\circled[1]{\tikz[baseline=(char.base)]{
            \node[shape=circle,draw,inner sep=2pt] (char) {#1};}}

\newcommand*\rcircled[1]{\tikz[baseline=(char.base)]{
            \node[shape=circle,draw,inner sep=2pt, color=red] (char) {#1};}}

\newenvironment{mybox}{\begin{tcolorbox}[sharp corners=all, frame empty]}{\end{tcolorbox}}


%\begin{document}
\setlength{\tabcolsep}{8pt}
\renewcommand{\arraystretch}{1.2}

\begin{tabular}[b]{c c c c c}
    \up & \down & \down & \up & \down \\
    \down & \up & \down & \down & \down \\
    \up & \up & \down & \up & \down \\
    \up & \down & \up & \up & \down
    %$\uparrow$ & $\uparrow$ & $\uparrow$ & $\uparrow$ & $\uparrow$
\end{tabular}
%\end{document}

        \end{minipage}
        \begin{minipage}[c]{0.2\textwidth} {\Huge $\stackrel{!}{\sim}$} \end{minipage}
        \begin{minipage}[c]{0.3\textwidth} {\Huge $e^{-\frac{E}{k_B T}}$} \end{minipage} \pause
    \item \textbf{Algorithm:}\\ \pause
            1.) Propose new state: Random spin flips!\\
            \vspace{2mm}
            \begin{minipage}{0.3\textwidth}
                {
\setlength{\tabcolsep}{8pt}
\renewcommand{\arraystretch}{1.2}

\begin{tabular}[b]{c c c c c}
    \up & \down & \down & \up & \up \\
    \down & \up & \rdown & \up & \up \\
    \up & \up & \up & \down & \down \\
\end{tabular}
}

            \end{minipage}
            \hfill
            \begin{minipage}[c]{0.2\textwidth} {\Huge $\Rightarrow$} \end{minipage}
            \begin{minipage}[c]{0.3\textwidth}
                {
\setlength{\tabcolsep}{8pt}
\renewcommand{\arraystretch}{1.2}

\begin{tabular}[b]{c c c c c}
    \up & \down & \down & \up & \up \\
    \down & \up & \rup & \up & \up \\
    \up & \up & \up & \down & \down \\
\end{tabular}
}

            \end{minipage}\\ \pause
            2.) Accept the new configuration: $p_a = e^{-\frac{\Delta E}{k_B T}}$
\end{itemize}
\end{frame}

\end{document}
